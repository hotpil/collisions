\documentclass[14pt]{beamer}
\usetheme{Copenhagen}
\usepackage[utf8]{inputenc}
\usepackage[russian]{babel}
\usepackage[OT1]{fontenc}
\usepackage{amsmath}
\usepackage{amsfonts}
\usepackage{amssymb}
\usepackage{graphicx}
\usepackage{bm}
\author{Маренков Е.Д.}
\title{Парные столкновения}
%\setbeamercovered{transparent} 
%\setbeamertemplate{navigation symbols}{} 
%\logo{} 
%\institute{} 
%\date{} 
%\subject{} 
\begin{document}

\begin{frame}
\titlepage
\end{frame}

%\begin{frame}
%\tableofcontents
%\end{frame}

\begin{frame}{Задача двух тел}

Чтобы определить движение двух тел, взаимодействующих с потенциалом $U(|\bm{r}_1-\bm{r}_2|)$, надо исходить из функции Лагранжа:
\begin{equation}
L=\frac{m\dot{\bm{r}_1}^2}{2} + \frac{m\dot{\bm{r}_1}^2}{2}
-U(r)
\end{equation}
$\bm{r}=\bm{r}_1=\bm{r}_2$
В системе центра масс:
\begin{equation}
L_c=\frac{\mu \bm{v}^2}{2} - U(r)
\end{equation}
$\mu=m_1m_2/(m_1+m_2)$ -- приведенная масса, $\bm{v}=\bm{v}_1-\bm{v}_2$ --- относительная скорость.

\end{frame}

\begin{frame}{Задача двух тел}

\alert{Вместо решения уравнений движения для двух тел можно решить задачу о движении одного тела с приведенной массой в центральном поле $U(r)$. }

Возврат в лабораторную систему:
\begin{equation}
\bm{r}_1=\frac{m_2}{m_1+m_2}\bm{r}, \quad 
\bm{r}_2=-\frac{m_1}{m_1+m_2}\bm{r}
\end{equation}

Зная $\bm{r}(t)$, можно определить по этим формулам $\bm{r}_1(t)$ и $\bm{r}_2(t)$


\end{frame}

\begin{frame}{Движение в центральном поле}

Момент импульса направлен перпендикулярно плоскости движения (ось $z$) и сохраняется: $M=M_z=m\bm{r}^2\dot{\phi}=const$. Полная энергия также сохраняется:
\begin{equation}
E=\frac{m\dot{r}^2}{2} + U_{eff}(r)
\label{eq:E}
\end{equation}
$U_{eff}=\frac{M^2}{2mr^2}+U(r)$ --- эффективная потенциальная энергия. 


\alert{Двухмерное движение свелось к одномерному в поле с эффективной потенциальной энергией}

\end{frame}

\begin{frame}{Движение в центральном поле}
Уравнение \eqref{eq:E} интегрируется в общем виде:
\begin{equation}
t=\int \frac{dr}{\sqrt{\frac{2}{m}[E-U(r)]-\frac{M^2}{m^2r^2}}} + const
\end{equation}

Из-за сохранения момента
\begin{equation*}
d\phi=\frac{M}{mr^2}dt
\end{equation*}

Поэтому
\begin{equation}
\phi=\int \frac{(M/r^2)dr}{\sqrt{2m[E-U(r)]-M^2/r^2}} + const
\end{equation}

\end{frame}



\end{document}